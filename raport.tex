\documentclass[]{report}

%polish
\usepackage[T1]{fontenc}
\usepackage[polish]{babel}
\usepackage[utf8]{inputenc}

%images
\usepackage{graphicx}
\usepackage{xcolor}
\definecolor{darkred}{RGB}{139,0,0}  % standard dark red


%uml diagram
\usepackage{pgf-umlcd}
\usepackage{geometry}
\usepackage{tikz}
%\geometry{margin=1cm}

\usepackage{multicol}


%place here
\usepackage{float}

%page numbering
\pagenumbering{arabic}

%fix section numbering
\renewcommand{\thesection}{\arabic{section}}
%\renewcommand\descriptionlabel[1]{$\cdot$ \textbf{#1}}


% Title Page
\title{Raport 2 Inżynieria Oprogramowania}
\author{Grupa 1}


\begin{document}
\maketitle
\tableofcontents
\newpage
\section{Wstęp}

W tym raporcie przedstawiony zostanie efekt pracy grupy 1 nad projektem z inżynierii oprogramowania.

Nasz zespół będzie budował aplikację mobilną, która w zasadzie będzie dwoma oddzielnymi podaplikacjami - aplikacją bibliotekarza i klienta biblioteki. Zależnie od rodzaju konta, które użytkownik się zaloguje, zostanie przeniesiony do odpowiedniej podaplikacji.

%--------------------------------
\newpage
\section{Specyfikacja}

%----------------------
\subsection{Funkcjonalność}

Aplikacja zawiera dwa zestawy dostępnych funkcjonalności - zestaw dla bibliotekarza i dla klienta. Część działań będzie wspólna dla obydwu użytkowników (funkcjonalność ogólna).

Funkcjonalności zostały podzielone na coraz to bardziej rozbudowane wersje aplikacji. Liczba w nawiasie określa numer wersji. Wersje porządkowane są rosnąco (1), (2), ... .

\subsubsection{Funkcjonalność ogólna:}

\begin{description}
    \item[(1) Wyświetlenie katalogu]  Wyświetlenie uporządkowanego alfabetycznie spisu księgozbioru.
    \item[(1) Wyszukanie książki]
    Wpisanie autora lub tytułu i wyświetlenie pozycji z katalogu do nich przypisanych.
\end{description}

\subsubsection{Funkcjonalność panelu bibliotekarza:}

\begin{description} 
    \item[(1) Dodanie publikacji] Wpisanie danych o publikacji i dodanie do księgozbioru.
    \item[(1) Dodanie instancji książki] Wybór w katalogu publikacji, do której będzie należeć dana książka, przypisanie kodu kreskowego instancji książki i dodanie do księgozbioru.
    \item[(3) Usunięcie publikacji] Wybór w katalogu publikacji i usunięcie jej z księgozbioru. 
    \item[(3) Usunięcie książki] Wybór w katalogu instancji książki i usunięcie jej z księgozbioru.
\end{description}

\subsubsection{Funkcjonalność panelu klienta:}

\begin{description}
    \item[(2) Wyświetlenie wypożyczeń] Wyświetlenie uporządkowanego chronologicznie spisu wypożyczeń.
    \item[(3) Wyświetlenie i edycja profilu] Wyświetlenie profilu użytkownika z wszystkimi danymi.
    \item[(1) Wypożyczenie książki] Zeskanowanie kodu na pożądanej instancji książki i dodanie jej do listy wypożyczeń, jeśli nie przekracza limitu.
    \item[(1) Zwrócenie książki] Zeskanowanie kodu na zwracanej instancji książki i usunięcie jej z listy wypożyczeń.
\end{description}


%----------------------
\subsection{Wybór technologii}

\subsubsection{Technologie frontendowe}

\textbf{React + React Native}

Są to popularne technologie umożliwiające tworzenie wieloplatformowych aplikacji, co jest niezwykle istotne przy próbie pogodzenia odrębnych potrzeb klienta biblioteki (mobilność, szybkość obsługi) i bibliotekarza (potrzeba masowego dodawania danych). Język JavaScript jest obecnie najpopularniejszym językiem na świecie, co znacznie powiększa dostępne rozwiązania i biblioteki.


\subsubsection{Technologie backendowe}

\textbf{Node.js + MySQL}

MySQL jest najpopularniejszym rozwiązaniem, jeśli chodzi o obsługę baz danych, i umożliwia kompleksowe i bezpieczne zarządzanie danymi po stronie serwera, Node.js umożliwia szybką i prostą obsługę serwera.

\subsubsection{Oprogramowanie do kolaboracji}

\textbf{freedcamp.com + Discord}

Przy wyborze oprogramowania do organizacji pracy czynnikiem decydującym były koszty z tym powiązane, co w przypadku aplikacji umożliwiających pracę w schemacie Kanban lub Scrum okazało się istotnym problemem. Udało się jednak znaleźć stronę spełniającą wszystkie nasze potrzeby.

Discord został wybrany ze względu na możliwość tworzenia kanałów — prostego segregowania wątków i grup zadaniowych, jednocześnie zachowując wszystkie konwersacje w jednym miejscu.

\vspace{3ex}

\noindent\textbf{ProjectLibre}

Użyjemy tego narzędzia do stworzenia wykresu Gaunta dla naszego projektu. Wybór konkretnie tego programu został nam narzucony przez laboratoria.


\subsubsection{Oprogramowanie do zarządzania wersjami}

\textbf{Git + GitHub}

Jest to oczywisty wybór. Git umożliwia zarządzanie wersjami i merging kodu pisanego przez wiele osób, jest to podstawowe narzędzie przy pracy nad większymi projektami informatycznymi.

%---------------------
\newpage
\section{Organizacja pracy nad projektem}
Z wykorzystaniem oprogramowania libre project sporządziliśmy wykres Gaunta naszego projektu. Jak na razie nie uwzględnia wszystkich zasobów, jakimi dysponujemy -  indywidualnych zdolności wszystkich członków zespołu. Mimo to diagram spełnił swoją rolę i pozwolił nam oszacować wstępną datę ukończenia projektu: 8 stycznia, co daje nam komfortowy zapas w razie nieprzewidzianych trudności z projektem.
\graphicspath{ {./diagram-gaunta.png}, {./gaunt-zbliżenie.png} }

\begin{figure}[H]
\includegraphics[width=\textwidth]{diagram-gaunta.png}
\textit{Wykres Gaunta dla zadań projektu}
\centering
\end{figure}

\begin{figure}[H]
\includegraphics[width=\textwidth]{gaunt-zbliżenie.png}
\textit{Zbliżenie wykresu Gaunta}
\centering
\end{figure}

\subsection{Podział na zespoły}

W celu organizacji pracy podzieliliśmy się na 2 główne zespoły - frontendowy i backendowy, w każdym został wytypowany manager mający na celu porządkować przepływ pracy odpowiedniego zespołu. Mianowialiśmy także 2 dodatkowe stanowiska - project manager i human resources.



\vspace{3ex}

\noindent\begin{minipage}[t]{0.32\textwidth}
    \subsubsection*{Frontend devs}

        Szymon Credo (\textbf{manager})
        
        Natalia Bardadyn
        
        Olimpia Dejko
        
        Szymon Doba
        
        Karol Dziuba
        
        Dawid Filipek

        Franek Gotkowski
        
        Olek Grzegrzułka
        
        Karol Jurkewicz

    \subsubsection*{Frontend testerzy}
        ...

\end{minipage}
\hfill
\begin{minipage}[t]{0.32\textwidth}
    \subsubsection*{Backend devs}
        Dawid Daraż (\textbf{manager})
        
        Adrian Cydejko
        
        Szymon Choiński
        
        Asia Dagil
        
        Michał Deptuła
        
        Mateusz Durka
        
        Danila Filipau
        
        Jakub Gankowski

    \subsubsection*{Backend testerzy}
        ...
    
\end{minipage}
\hfill
\begin{minipage}[t]{0.32\textwidth}
    \subsubsection*{Project manager}
        Franek Gotkowski
    \subsubsection*{HR}
        Piotr Czarnocki
    
\end{minipage}

\subsection{Szkolenia}

W ramach projektu przeprowadzamy także szkolenia zespołu, mające na celu wdrożenie wszystkich do obsługi potrzebnych narzędzi albo zaznajomienie się z używanymi technologiami. Poniżej zamieszczamy listę odbytych i planowanych szkoleń.

\begin{description}
    \item[23/10/2025 "Wstęp do serwerów i wysyłanie zapytań"]
    
        prowadzący: Franek Gotkowski
        
    \item[30/10/2025 "Wstęp do Githuba"]
    
        prowadząca: Asia Dagil

        Instalacja Gita i założenie kont na Githubie. Stworzenie repozytorium projektu i dodanie swojego brancha przez każdego członka zespołu. Przejście przez przykładowy workflow.

    \item[06/11/2025 "Wstęp do Reacta"]
    
        prowadzący: Franek Gotkowski
\end{description}

% ---------------------------------------

\newpage
\section{Model UML}

Wstępny diagram przedstawiając model bazy danych w MySQL.
% https://www.w3schools.com/mysql/mysql_datatypes.asp
\vspace{3ex}

% to do

\begin{tikzpicture}
    \tikzstyle{every node}=[font=\small]

    \begin{class}[text width=4cm]{publisher}{-5.5,0}
        \attribute{publisher\_id \hfill: INT(10)}
        \attribute{publisher \hfill: VARCHAR(40)}
    \end{class}

    \begin{class}[text width=4cm]{book-author}{-5.5,-2.3}
        \attribute{author\_id \hfill: INT(10)}        
        \attribute{book\_id \hfill: INT(10)}
    \end{class}

    \begin{class}[text width=4cm]{author}{-5.5,-4.8}
        \attribute{author\_id \hfill: INT(10)}
        \attribute{name \hfill: VARCHAR(20)}
        \attribute{surname \hfill: VARCHAR(20)}
    \end{class}

    \begin{class}[text width=4cm]{book-genre}{-5.5,-7.6}
        \attribute{genre\_id \hfill: INT(10)}
        \attribute{book\_id \hfill: INT(10)}
    \end{class}

    \begin{class}[text width=4cm]{genre}{-5.5,-10.2}
        \attribute{genre\_id \hfill: INT(10)}
        \attribute{genre \hfill: VARCHAR(20)}
    \end{class}
    
    \begin{class}[text width=4cm]{book-key-word}{-5.5,-12.5}
        \attribute{key\_word\_id \hfill: INT(10)}
        \attribute{book\_id \hfill: INT(10)}
    \end{class}

    \begin{class}[text width=4cm]{key-word}{-5.5,-15.1}
        \attribute{key\_word\_id \hfill: INT(10)}
        \attribute{key\_word \hfill: VARCHAR(20)}
    \end{class}

    %-------------------------------------

    \begin{class}[text width=4cm]{book}{0,0}
        \attribute{book\_id \hfill: INT(10)}
        \attribute{publisher\_id \hfill: INT(10)}
        \attribute{title \hfill: VARCHAR(50)}
        \attribute{publish\_year \hfill: YEAR}
        \attribute{isbn \hfill: VARCHAR(13)}
        \attribute{available\_inst \hfill: INT(10)}
    \end{class}

    \begin{class}[text width=4cm]{instance}{0,-4.2}
        \attribute{instance\_id \hfill: INT(10)}
        \attribute{book\_id \hfill: INT(10)}
        \attribute{available \hfill: BOOLEAN}
    \end{class}


    \begin{class}[text width=4cm]{rental}{0,-7}
        \attribute{rental\_id \hfill: INT(10)}
        \attribute{instance\_id \hfill: INT(10)}
        \attribute{user\_id \hfill: SMALLINT(5)}
        \attribute{extensions \hfill: BIT(1)}
        \attribute{borrow\_date \hfill: DATE}
        \attribute{return\_date \hfill: DATE}
    \end{class}

    
    
    \begin{class}[text width=4cm]{user}{0, -11.6}
        \attribute{user\_id \hfill: INT(10)}
        \attribute{address\_id \hfill: INT(10)} 
        \attribute{card\_id \hfill: INT(10)} 
        \attribute{personal\_info\_id \hfill: INT(10)} 
    \end{class}

    \begin{class}[text width=4cm]{worker}{0, -15.2}
        \attribute{worker\_id \hfill: INT(10)}
        \attribute{personal\_info\_id \hfill: INT(10)} 
    \end{class}

    % --------------------------------------------
    \begin{class}[text width=4cm]{address}{5.5, -5.6}
        \attribute{address\_id \hfill: INT(10)}
        \attribute{user\_id \hfill: INT(10)}
        \attribute{city \hfill: VARCHAR(16)}
        \attribute{postcode \hfill: VARCHAR(5)}
        \attribute{street \hfill: VARCHAR(20)}
        \attribute{street\_no \hfill: VARCHAR(4)}
        \attribute{flat\_no \hfill: VARCHAR(4)}
    \end{class}
    
    \begin{class}[text width=4cm]{card}{5.5, -10.1}
        \attribute{card\_id \hfill: INT(10)}
        \attribute{user\_id \hfill: INT(10)}
        \attribute{number \hfill: VARCHAR(16)}
        \attribute{exp \hfill: VARCHAR(4)}
        \attribute{cvv \hfill: VARCHAR(4)}
    \end{class}

    \begin{class}[text width=4cm]{personal-info}{5.5, -13.6}
        \attribute{personal\_info\_id \hfill: INT(10)}
        \attribute{name \hfill: VARCHAR(30)}
        \attribute{surname \hfill: VARCHAR(30)}
        \attribute{phone \hfill: VARCHAR(20)}
        \attribute{email \hfill: VARCHAR(50)}
        \attribute{password \hfill: VARCHAR(20)}
    \end{class}

    %---------------------------------------
    %\association{worker}{}{1}{personal-info}{}{*}

    \draw (-3.38,-0.7) node[above, xshift=5pt]{\small 1} -- (-2.12,-0.7) node[above, xshift=-4pt]{\small *}; % publisher-book
    
    \draw[] (-3.38,-3.2) node[above, xshift=5pt]{\small *}-- (-3, -3.2) -- (-3, -1.3) -- (-2.12,-1.3) node[above, xshift=-4pt]{\small 1}; % book author - book

    
    \draw (-3.38,-8.5) node[above, xshift=5pt]{\small *} -- (-2.7, -8.5) -- (-2.7, -1.8) -- (-2.12,-1.8)node[above, xshift=-4pt]{\small 1}; % book genre - book

    \draw (-3.38,-13.5) node[above, xshift=5pt]{\small *} -- (-2.4, -13.5) -- (-2.4, -2.4) -- (-2.12,-2.4) node[above, xshift=-4pt]{\small 1}; % book key word - book

    
    %\association{publisher}{}{1}{book}{}{*}
    \association{author}{}{1}{book-author}{}{*}
    %\association{book author}{}{*}{book}{}{1}
    %\association{book genre}{}{1}{book}{}{*}
    \association{book-genre}{}{*}{genre}{}{1}
    \association{key-word}{}{1}{book-key-word}{}{*}
    %\association{book key word}{}{1}{book}{}{*}
    \association{book}{}{1}{instance}{}{*}
    
    % book_copy <-> rental
    \association{instance}{}{1}{rental}{}{*}
    
    % rental <-> user
    \association{user}{}{1}{rental}{}{*}

    \draw (2.12,-16) node[above, xshift=5pt]{\small 1} -- (3.38,-16) node[above, xshift=-4pt]{\small 1}; % worker - info

    \draw (2.12,-13.4) node[above, xshift=5pt]{\small 1} -- (2.7,-13.4) -- (2.7,-15.2) -- (3.38,-15.2) node[above, xshift=-4pt]{\small 1}; % user - info

    \draw (2.12,-12.6) node[above, xshift=5pt]{\small 1} -- (3.38,-12.6) node[above, xshift=-4pt]{\small 1}; % user - card

    \draw (2.12,-12.1) node[above, xshift=5pt]{\small 1} -- (2.7,-12.1) -- (2.7,-8) -- (3.38,-8) node[above, xshift=-4pt]{\small 1}; % user - address

    %\association{user}{}{1}{address}{}{1}
    %\association{user}{}{*}{status}{}{1}
    %\association{user}{}{1}{card}{}{1}

    
\end{tikzpicture}


%--------------------------------
%\newpage
%\section{Scenariusze testowe}


%--------------------------------
\newpage
\section{Aplikacja}


%--------------------------------
\newpage
\section{GUI}


%--------------------------------
%\newpage
%\section{Doxygen}


%--------------------------------
%\newpage
%\section{Instrukcja obsługi}


%--------------------------------
%\newpage
%\section{Testy}

%---------------------
%\subsection{Protokoły testowe}
%---------------------
%\subsection{Rezultaty testów}


\end{document}          
